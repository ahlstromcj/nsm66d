%-------------------------------------------------------------------------------
% nsm66-reference-guide
%-------------------------------------------------------------------------------
%
% \file        nsm66-reference-guide.tex
% \library     Documents
% \author      Chris Ahlstrom
% \date        2025-01-28
% \update      2025-01-28
% \version     $Revision$
% \license     $XPC_GPL_LICENSE$
%
%     This document provides LaTeX documentation for the nsm66 library.
%
%-------------------------------------------------------------------------------

\documentclass[
 11pt,
 twoside,
 a4paper,
 final                                 % versus draft
]{article}

\input{tex/docs-structure}             % specifies document structure and layout

\usepackage{fancyhdr}
\pagestyle{fancy}
\fancyhead{}
\fancyfoot{}
\fancyheadoffset{0.005\textwidth}
\lhead{Nsm66 Library Guide}
\chead{}
\rhead{Developer Guide}
\lfoot{}
\cfoot{\thepage}
\rfoot{}

% Removes the many "headheight is too small" warnings.

\setlength{\headheight}{14.0pt}

\makeindex

\begin{document}

\title{Nsm66 Developer Guide 0.1.0}
\author{Chris Ahlstrom \\
   (\texttt{ahlstromcj@gmail.com})}
\date{\today}
\maketitle

\begin{figure}[H]
   \centering 
   \includegraphics[scale=0.25]{nsm66.png}
   \caption*{Nsm66 Logo}
\end{figure}

\clearpage                             % moves Contents to next page

\tableofcontents
\listoffigures                         % print the list of figures
\listoftables                          % print the list of tables

% changes the paragraph style to remove indenting and put a line between each
% paragraph.  this could be moved up into the preamble, but then would
% affect the spacing of the toc and lof, lot noted above.

\setlength{\parindent}{2em}
\setlength{\parskip}{1ex plus 0.5ex minus 0.2ex}

\rhead{\rightmark}         % shows section number and section name

\section{Introduction}
\label{sec:introduction}

   The \textsl{Nsm66} library reworks some of the fundamental code
   from ...

\subsection{Naming Conventions}
\label{subsec:introduction_conventions}

   \textsl{Nsm66} uses some conventions for naming things in this
   document.

   \begin{itemize}
      \item \texttt{\$prefix}. The base location for installation of
         the application and its ancillary data files on
         \textsl{UNIX/Linux/BSD}:
         \begin{itemize}
            \item \texttt{/usr/}
            \item \texttt{/usr/local/}
         \end{itemize}
      \item \texttt{\$winprefix}. The base location for installation of
         the application and its ancillary data files on \textsl{Windows}.
         \begin{itemize}
            \item \texttt{C:/Program Files/}
            \item \texttt{C:/Program Files (x86)/}
         \end{itemize}
      \item \texttt{\$home}. The location of the user's configuration files.
         Not to be confused with \texttt{\$HOME}, this is
         the standard location for configuration files.
         On a UNIX-style system, it would be \linebreak
         \texttt{\$HOME/.config/appname}.
         The files would be put into a \texttt{po} subdirectory here.
      \item \texttt{\$winhome}. This location is different for
         \textsl{Windows}:
         \texttt{C:/Users/user/AppData/Local/PACKAGE}.
   \end{itemize}

\subsection{Future Work}
\label{subsec:introduction_future}

   \begin{itemize}
      \item Hammer on this code in \textsl{Windows}.
   \end{itemize}

%----------------------------------------------------------------------------
% Additional Chapters
%----------------------------------------------------------------------------

% \input{tex/cfg}

\section{Summary}
\label{sec:summary}

   Contact: If you have ideas about \textsl{Nsm66} or a bug report,
   please email us (at \url{mailto:ahlstromcj@gmail.com}).

% References

%-------------------------------------------------------------------------------
% references
%-------------------------------------------------------------------------------
%
% \file        references.tex
% \library     Documents
% \author      Chris Ahlstrom
% \date        2025-01-28
% \update      2025-01-28
% \version     $Revision$
% \license     $XPC_GPL_LICENSE$
%
%     Provides the References section of the Nsm66 manual. Rather
%     than use the bibtex package, our small set of references uses a
%     simpler method.
%
%-------------------------------------------------------------------------------

\section{References}
\label{sec:references}

   The \textsl{Nsm66} references list.

{\RaggedRight
\begin{thebibliography}{99}

   \bibitem{deque}
   Twinkle Sharma.
   \emph{Deque in C++}
   \url{https://www.scaler.com/topics/cpp/deque-in-cpp/}.
   2024.

\end{thebibliography}
}

%-------------------------------------------------------------------------------
% vim: ts=3 sw=3 et ft=tex
%-------------------------------------------------------------------------------


\printindex

\end{document}

%-------------------------------------------------------------------------------
% vim: ts=3 sw=3 et ft=tex
%-------------------------------------------------------------------------------
